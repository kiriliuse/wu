\documentclass{article}

\usepackage{amsmath, amsthm, amssymb, amsfonts}
\usepackage{thmtools}
\usepackage{graphicx}
\usepackage{setspace}
\usepackage{geometry}
\usepackage{float}
\usepackage{hyperref}
\hypersetup{pdfborder=0 0 0}
\usepackage[utf8]{inputenc}
\usepackage[english]{babel}
\usepackage[dvipsnames]{xcolor}
\usepackage{tcolorbox}

\usepackage[
backend=biber,
style=numeric,
sorting=ynt
]{biblatex}
\addbibresource{References.bib}

\colorlet{LightGray}{White!90!Periwinkle}
\colorlet{LightOrange}{Orange!15}
\colorlet{LightGreen}{Green!15}

\newcommand{\HRule}[1]{\rule{\linewidth}{#1}}

\setstretch{1.2}
\geometry{
    textheight=9in,
    textwidth=5.5in,
    top=1in,
    headheight=12pt,
    headsep=25pt,
    footskip=30pt
}

% ------------------------------------------------------------------------------


\begin{document}

% ------------------------------------------------------------------------------
% Cover Page and ToC
% ------------------------------------------------------------------------------

\title{ \normalsize \textsc{}
		\\ [2.0cm]
		\HRule{1.5pt} \\
		\LARGE \textbf{\uppercase{Case Study Too Good To Go}
		\HRule{2.0pt} \\ [0.6cm] \LARGE{Aspects of a successful and sustainable business venture} \vspace*{10\baselineskip}}
		}
\date{}
\author{\textbf{Benjamin Meixner} \\ 
		\textbf{Faraz Muhammad} \\
		\textbf{Pavel Lazarev} \\
		\textbf{Matthäus Engelbrecht}
		}

\maketitle
\newpage

\tableofcontents
\newpage

% ------------------------------------------------------------------------------

\section{TooGoodToGo Where?}

Over 30\% of food is lost or wasted each year.\cite{waste} This number is even
more striking, given the large number of hungry people in the world. Wasted
food is not only inefficient, it’s a social justice issue. Founded in 2016 in
Copenhagen, Too Good To Go Holding APS (TGTG) helps everyday people fight food
waste in their local communities by connecting them to restaurants and grocery
stores with surplus meals and ingredients through an app (marketplace). Once
registered, users can choose from their favorite local spots to pick up meals
like sushi, or ingredients like apples and avocados for a third of the normal
price. This presents a win-win-win solution in which consumers get great food,
store owners no longer throw away delicious surplus and most importantly
directly address the impact it has on the environment and climate. TGTG
monetizes this process by charging a small fee to the food business for every
meal saved as well as charging food outlets an annual administrative fee for
being registered on their platform.

The mission of this company is to inspire and empower everyone to fight food
waste together. To this end, the team initiates projects and builds
partnerships with businesses, households, schools and governments to create
real change in legislation against, educate about and practice reduction in
food waste.\cite{impact}

With the company now on track to profitability, it is launching the next phase
of its development headlined “Impact at Scale”. This stage will enable the
business to save food from going to waste at a much bigger scale than at
present. This is observed through the acquisition of the French SaaS (Software
as a service) company CodaBene with their main product ‘FoodMemo’. FoodMemo is
a 2-in-1 solution to detect short-dated food products and optimize their
end-of-life. With FoodMemo as part of TGTG’s offering, they can service medium
and larger size retailers across our markets with an even stronger package. For
2023, TGTG is expected to see a continued solid double digit revenue growth –
supported by both the marketplace, FoodMemo and additional product development.

\section{Market Orientation}

The largest base pillar of a service like TGTG is strong marketing. Since the
service is practically capped with the prices that they can earn with a single
meal saved, they can only meaningfully increase revenue by saving more meals.
TooGoodToGo has increased its annual saved meal count by 50\% and reached 20
million additional users within 2022 \cite{impact}. But how did they manage
such a marketing feat?

TGTG is in a challenging situation, since they cannot invest huge amounts of
capital into marketing. Therefore, they have to work with low-budget marketing
channels such as stickers and branded paper bags. These are low-cost and still
promote the brand at key points in the decision making of customers. Another
way in which they excel is in media coverage. The company and its services
have received more than 25.000 media occurrences \cite{impact}. A major part
of their marketing channel management is mouth-to-mouth marketing. Since
perfectly good food at a fraction the price is a topic which is interesting to
virtually everyone, and this mode of marketing works exceptionally well for
TGTG. According to semi-structured interviews, social, functional, and
emotional values are the success factors for the TGTG app to accomplish its
social missions of reducing food waste and CO2 emissions and allowing everyone
to access quality food at an affordable price \cite{vo-thann}.

One more interesting aspect is the diversification of products in different
countries. Since they operate in 17 countries \cite{impact} and each country is
different, they need a more diverse product portfolio. Not only the
\textbf{Magic Bags} where most meals are saved directly from restaurants are a
success. Also the recent \textbf{Magic Parcels} target a different part of the
food supply chain, in which they aim to reduce the waste in the production or
preparation of food. To complete the product range they have 2 services
targeted at resellers. \textbf{FoodMemo} is handling out-of-date times of
various products and helps businesses with forming better decisions about their
stock of foods while the \textbf{often good for longer} stickers signal to
customers that this product can be consumed oftentimes beyond the run-off date.
\cite{impact}

The ongoing growth of TGTG is likely to continue since there is still a great
potential in this market. Food waste is not just a pressing issue of society as
a whole but also a lucrative business endeavor for all participants. First off,
there are still a lot more countries to expand to. France alone accounts for
almost 60\% of sales while all other 16 countries share 40\% of meals saved.
Also, more stores and restaurants are continuously signing up and therefore
there is still great potential even within well-established countries.

\section{Personell}

TGTG is a promising startup and is steadily growing. This can be proven by
their employee numbers which rose by 19.5\% in the last year. They are not just
growing, but also paying their employees better than before. This can be seen
by their operating loss, which decreased by 48\% and of that 89\% go to
employees. In addition to that, TGTG offers a warrant program to retain certain
key employees. In 2022, 174,330 warrants were granted under the program. That's
more than triple the number of warrants granted in 2021 (50,330). Those
warrants are exercisable within 1-10 years as of grant date. The right to vest
them is subject to continued employment. As of 31 December 2022, 305,680
warrants were outstanding but not all vested to be exercised. \cite{impact}

They also place a big emphasis on Equality in the Workplace. According to their
Impact Report they consist of 53.3\% women, have an LGBTQIA+ quota of 13.3\%,
disability quota of 6.2\%, and employ 13\% of ethnic minorities. Their
Gender-specific salary falloff is at 6.04\%. Even though this is far below the
European average of 13\%, TGTG still comments that they want to bring this
number down to zero. \cite{impact}

To further improve their diversity within employees, they created their plans
for Diversity, Equity and Inclusion (DE\&I). In these plans, they set the
following Goals until 2025: \begin{itemize} \item TGTG wants at least two
			people from an underrepresented group on their Board. \item They want to
				increase the percentage of women in management positions from 38\% to
				50\% and the percentage of ethnically diverse groups from 0\% to 10\%
			\item They intend to increase the ratio of ethnic groups in the workplace
				from 9\% to at least 15\% \item TGTG wants an increase in people
				without a University diploma from 3.2\% to 15\% \cite{impact}
\end{itemize}

TGTG also introduced so-called "Employee Resource Groups". These are supposed
to support integration and representation of the following marginalized groups:
Women, LGBTQIA+, People of Color and People with a disability. \cite{impact}

They also have a program to motivate their employees to work at  local food
banks. Within this program, they 'gift' their employees four paid working days.
757 employees took advantage of this program and accumulated 3000 hours in
2022, and 5700 in total. \cite{impact}

On online rating platforms flexible working hours and good teams get praised:
however, there is also some criticism \cite{kanunugermany}. TGTG has a 4 star
rating on Kununu for the DACH-area and Glassdoor for the US (both 3.8 out of
5). Some reviews mention stress, high turnover, low pay, limited growth
prospects, and management issues. 
\cite{kanunuus} \cite{kanunuaustria} \cite{kanunuswitzerland}

\section{Social Responsibility}

The practice of inclusion and diversity in their workforce can also be seen as
a fulfillment of their corporate social responsibility (CSR) with regards to
business ethics. The importance put into fairness, compassion, transparency and
environmental concern is reflected in their Code of Ethics and their zero tolerance
policies towards harassment.\cite{impact} This will not only motivate their
workforce but strengthen the company’s vision and image. It may result in
attracting investment and retaining existing customers as well as bringing new
ones.

With relevance to stakeholder management, everyone in the supply chain is
winning. The suppliers, or in this case businesses selling surplus food, have
an increase in revenue which would otherwise be a wasteful expense (which is
almost 79 million meals just in 2022 alone). Customers across 17 markets are
buying the same food at a cheaper price than they would have gotten by buying
through standard means. 

B Corp certified companies are verified by the non-profit network B Lab to meet
high standards of social and environmental performance, transparency and
accountability. \cite{blab} In 2019 TGTG became a B Corp to verify the high
standards of social and environmental impact in how the business is run, and to
show its commitment to delivering business in a way where profit and purpose
are interlinked. Being a B Corp means they have changed their Articles of
Association to ensure they have a significant positive impact on society and
the environment as a whole. Furthermore, TGTG was named “Best in the World” in
the governance category twice in a row. \cite{impact}

TGTG being one of the leading examples of fighting food waste is not only
looking at continuously improving and developing itself sustainably, but also
contributing in the movement against food waste on a global scale. At 2022’s
COP27, the UN’s annual climate conference,TGTG’s policy team took part in a
panel discussion on ‘Why the climate crisis demands food waste regulation’. A
full program of ‘food systems’ events was run at COP27, finally acknowledging
the importance of food in the climate crisis.

TGTG has joined the ‘123 Pledge’ \cite{unep}. A call to arms challenging
governments, companies, and institutions to commit to concrete steps towards
reducing the climate impact of food waste. TGTG has committed to, among other
things, encouraging and supporting governments in ten countries to shape and
improve food waste policy measures. Being a member of the EU Platform on Food
Loss and Waste, TGTG has assisted in drafting regulations and proposals in
Spain, Denmark, France and Austria as of 2022.

\section{Liquidity and Financial Resources}

The financial statement, published in the Annual Report (2022), reflects the
growth and development of TGTG and can thus be considered promising for the
company and its customers.

Launched in 2016, it took TGTG six years to reach the first 100 million meals
saved; the next 100 million meals were saved 5 times faster, in under 1.5 years
\cite{milestones}. The company’s continued extension is reflected not only in
the popularity of the app (7th most downloaded app worldwide in the Food \&
Drink category), acquirement of CodaBene (French company with a solution to
detect short-dated products and optimize their end-of-life) or the number of
business customers (130,000 monthly active stores), but also in the increased
turnover (revenue) (+53.7\%) in all geographic regions, where TGTG operates, in
comparison to the year of 2021. 

Accordingly, gross profit grew by more than 50\% (62.3 $\rightarrow$ 95.2 mln EUR), while
the cost of sales (goods sold to the customers and the payment fees related to
the transactions on the marketplace) decreased by 40.2\%, which demonstrates
improved efficiency.

Operating and net loss could be explained by the relatively high fixed costs,
and their main component is the cost of employees, whose number increased by
19.6\% in 2022. At the current phase of the company growth, these numbers are
justifiable by the need to hire and retain skilled personnel, that will allow
improving positioning and win new customers in the geographical areas to which
TGTG expanded in 2021. Another positive sign is diminished external (mainly
marketing) expenses, which were reduced by almost 20\% in 2022.

Overall, there was 35\% less operating loss (EBITDA, -28.6 vs -44.1 mln EUR),
27.7\% less net loss (-37.6 vs -52.1 mln EUR) as well as improvement in ROCE
metric in 2022, compared with 2021, which could signal the (near) end of
COVID-19 negative effects (TGTG does not use delivery services) and the
beginning of market penetration in the ‘new’ countries. Moreover, the company
managed to make a positive EBITDA before special items and share-based payment
in Q4 2022.

Cash flow statement shows a positive cash balance (liquidity) with a slightly
negative operating cash balance (albeit much less so than in the year before)
and reflects the ongoing investments. Inflow of cash exceeding 50 mln EUR from
existing and new investors in 2022 mirrors their belief in the business idea
and future of TGTG.

\section{Cost and Risk Awareness}

The management of TGTG recognizes the value of having a resilient business
model, product/marketing, and personnel agility for overcoming existing and
possible challenges. Agility of their business itself is reflected in the
revised strategy to assess progress on a monthly (rather than annual) basis and
a new operating model with four main regions ‘in focus’ and global support. The
decision to stay within the same operating area and not no launch new markets
in the year of 2022 comprises a rational strategy to avoid unnecessary risk and
prevent excessive loss from overinvesting while exploiting the market potential
in the current operating areas. Understandably, partner-related investments of
the company aim to drive supply consistency and volume, which could help to
keep both businesses and customers on the platform.

In their own words, TGTG strengthened the focus on Enterprise Risk Management
in 2022, with active engagement of senior leaders in identification and
(re-)validation of risks integrated in business operations and strategic
planning. Regular review of the existing risks, their potential impact, and
countermeasures is part of the routine in the company as part of the Risk
Management and Internal Control Framework. Most crucial risk subjects which
partially correspond to risks perceived by consumers \cite{fraccascia} and ways
to mitigate them were highlighted in the Annual Report \cite{impact}. Based on
the above, one can conclude that TGTG appreciates the costs and risks in
accordance with their relevance and importance, and this appreciation is
incorporated in their realistic business strategy.

\section{Conclusion}

Overall, the business idea of TGTG is based around sustainability and actively
promotes tackling food waste through the marketplace itself  [200,000 CO2
emissions prevented in 2022] and ongoing initiatives mentioned above, which is
why we did not include a separate section on this aspect of the business and
decided to evaluate other characteristics instead. Although there is room for
improvement and still some way to go before the company makes a sizable profit,
TGTG already stood the test of time for a start-up, not only having survived
the pandemic and being in business for 8 years, but also expanding, enjoying
good publicity and a growing number of customers. The success of TGTG and its
potential to be a role model for aspiring sustainable businesses was recognised
in 2022 when it was included in the list of 100 Most Influential Companies
compiled annually by the Time Magazine \cite{reduce}. We hope it is just the beginning
of the story for TGTG to become an even ‘better, kinder, and more effective
business’, in the words of its CEO, Mette Lykke.  

\newpage

% ------------------------------------------------------------------------------
% Reference and Cited Works
% ------------------------------------------------------------------------------

\printbibliography

% ------------------------------------------------------------------------------

\end{document}
